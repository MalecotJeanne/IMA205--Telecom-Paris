\documentclass[12pt,a4paper]{article}

\usepackage{geometry}
\usepackage{amsfonts}
\usepackage{amsmath}

\geometry{hmargin=2.5cm,vmargin=1.5cm}

\title{TP : Introduction Supervised Learning}
\author{Jeanne Malécot}
\date{}

\begin{document}

\maketitle

\section{OLS}
Since $\tilde{\beta}$ is unbiased : 

\begin{align} 
    \beta & =\mathbb{E}(\tilde{\beta})              \\
          & = \mathbb{E}(Hy) + \mathbb{E}(Dy)       \\  
          & = \mathbb{E}(\beta ^*) + D\mathbb{E}(y) \\
          & = \beta  + D\mathbb{E}(y)
\end{align}
Then we have $DX\beta = 0$, and finally :

$$ 
    \boxed{DX = 0}
$$

Now we compute :
\begin{align} 
    Var(\tilde{\beta}) & = Var(Cy)                                                  \\
                       & = \sigma^2(H+D)(H^T+D^T)                                   \\  
                       & = \sigma^2HH^T + \sigma^2DD^T + \sigma^2DH^T +\sigma^2HD^T \\
                       & = \sigma^2HH^T + \sigma^2DD^T 
\end{align}
And $ Var(\beta^*) = \sigma^2 HH^T $.   
So we have
$$
    Var(\tilde{\beta}) = Var(\beta^*) + \sigma^2DD^T    
$$
And finally : 

$$
    \boxed{Var(\tilde{\beta}) > Var(\beta^*) }
$$

\section{Ridge Regression}

We have $\beta^*_{ridge} = (\lambda I_d + X^TX)^{-1}X^TY$. 

\begin{align} 
    \mathbb{E}[\beta^*_{ridge}] & = \mathbb{E}[(\lambda I_d + X^TX)^{-1}X^Ty]                                                        \\
                                & = (\lambda I_d + X^TX)^{-1}X^TX\theta                                                              \\ 
                                & = (\lambda I_d + X^TX)^{-1}(\lambda I_d + X^TX)\theta - \lambda I_d(\lambda I_d + X^TX)^{-1}\theta \\
                                & = \theta - \underbrace{\lambda I_d(\lambda I_d + X^TX)^{-1}\theta}_{bias}
\end{align}

$X = UDV^T$, so :

\begin{align} 
    \beta^*_{ridge} & = (\lambda I_d + X^TX)^{-1}X^T      \\
                    & = (\lambda I_d + VD^2V^T)^{-1}VDU^T \\
                    & = (V(\lambda +D^2)V^T)^{-1}VDU^T    \\    
                    & = V(D^2 + \lambda)^{-1}DU^T
\end{align}

$$ \boxed{ \beta^*_{ridge}= V(D^2 + \lambda)^{-1}DU^T} $$

\begin{align}
    Var(\beta^*_{ridge}) & = \mathbb{E}[(V(D^2 + \lambda I_d)^{-1}DU^T\epsilon)(V(D^2 + \lambda I_d)^{-1}DU^T\epsilon)^T] \\
                         & = \sigma^2V(D^2 + \lambda I_d)^{-1}D^2(D^2 + \lambda I_d)^{-1}V^T 
\end{align}

With $D = diag(d_i)$ : 
$$
    Var(\beta^*_{ridge}) = \sigma^2V diag(\frac{d_i^2}{(d_i^2 + \lambda)^2})V^T
$$

So : $ Var(\beta^*_{ridge}) -  Var(\beta^*_{OLS}) = \sigma^2Vdiag(\frac{1}{d_i^2}(\frac{1}{(1+ \frac{\lambda}{d_i^2})^2} - 1))V^T$.

If $ \lambda \geq 0 $ :  

$$
    \frac{1}{d_i^2}( \frac{1}{(1+ \frac{\lambda}{d_i^2})^2} - 1) \leq 0
$$

And then :
$$
    \boxed{Var(\beta^*_{ridge}) \leq  Var(\beta^*_{OLS})}
$$


We have :
$$
    \lim_{\lambda \to +\infty} Var(\beta^*_{ridge}) = 0 $$
$$
    \lim_{\lambda \to 0} Var(\beta^*_{ridge})  = \sigma^2(X^TX)^{-1}
$$

Since $ \mathbb{E}[\beta^*_{ridge}] =  V diag(\frac{d_i^2}{(d_i^2 + \lambda)})V^T\theta $ :

$$
    \boxed{\lim_{\lambda \to +\infty} \mathbb{E}[\beta^*_{ridge}] = 0} 
$$
$$
\boxed{\lim_{\lambda \to 0} \mathbb{E}[\beta^*_{ridge}] =  \theta}
$$

\end{document}